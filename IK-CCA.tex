\section{IK-CCA implies Eval-Security}
\label{sec:IND-CCPKA}
In this section, we propose a symmetric key analog of the IK-CCA security introduced by Bellare et. al. in \cite[Definition 1]{BBDP01keyprivate}
We then show that IK-CCA security implies the Eval-Security \cref{defn:AE-eval-security} needed for the security of our system.

Recall that an AE scheme $\Pi_{\ae}$ consists of algorithms $(\gen, \enc, \dec)$ with syntax defined in \cref{subsec:AE}. 

\begin{definition}[IK-CCA]
\label{defn:AE-IK-CCA}
Consider the following distinguishing experiment, where $N = N(\lambda)$ is polynomial in $\lambda$.
\begin{figure}[ht!]
\begin{framed}
Distinguishing Game $\mathbf{Exp}_{\mathsf{IK-CCA}}^{\cA}$
\begin{enumerate}
    \item $\sk_0, \sk_1 \leftarrow \gen(1^{\lambda}).$
    \item $m_0, m_1 \leftarrow \cA^{\Enc(\{\sk_{i}\}, \cdot), \Dec(\{\sk_{i}\}, \cdot)}(1^{\lambda}, kx_{i}^P).$
    \item $b \leftarrow U(\{0, 1\})$.
    \item $\ct \leftarrow \Enc(\sk_b, m_b).$
    \item $b' \leftarrow \cA^{\Enc(\{\sk_{i}\}, \cdot), \Dec(\{\sk_{i}\}, \cdot)}(\ct)$.
\end{enumerate}
\end{framed}
\caption{Distinguishing Game for IK-CCA security}
\label{expr:AE-IKCCA-Distinguish}
\end{figure}

Let $\ct_{\query}$ denote the queries $\cA$ sent to both decryption oracles on line (5).  We define the output of the experiment as $1$ if both $b' = b$ and $\ct \notin \ct_{\query}$, and $0$ otherwise. 

Then we say the key exchange plus symmetric key system $\Pi_{\ae}$ is \textbf{IK-CCA secure} if for any p.p.t with oracle adversary $\cA^O$, we have
$$\PP(\mathbf{Exp}^{\mathsf{IK-CCA}}_{\cA} = 1) \leq \frac{1}{2} + \negl(\lambda).$$
\end{definition}

Let $\Pi_{\ae}$ be IK-CCA. We now show that $\Pi_{\ae}$ is Eval-secure. We recall the relevant definitions, where the simulator simply outputs $\Enc(\sk_{0}, 0^{L})$ for $\sk_0 \leftarrow \gen(1^{\lambda})$.

\begin{figure}[ht!]
\begin{framed}
\textbf{Real World Experiment $\Real^{\cA}_{\Eval}$}
\begin{enumerate}
    \item For $i$ from $1$ to $N$, $\sk_i \leftarrow \gen(1^{\lambda}).$
    \item For $r$ from $1$ to $R$
    \begin{enumerate}
        \item $i, \ct, \arg_p \leftarrow \cA(1^{\lambda}).$
        \item $\ct^{0}_r \leftarrow \Eval_f(\sk_i, \ct, \arg_p)$.
        \item $\cA$ stores $\ct^{0}_r$.
    \end{enumerate}
\end{enumerate}
\textbf{Ideal World Experiment $\Ideal^{\cA}_{\Eval}$}
\begin{enumerate}
    \item For $i$ from $1$ to $N$, $\sk_i \leftarrow \gen(1^{\lambda}).$
    \item For $r$ from $1$ to $R$
    \begin{enumerate}
        \item $i, \ct, \arg_p \leftarrow \cA(1^{\lambda}).$
        \item $\sk_0 \leftarrow \gen(1^{\lambda}), \ct^{1}_r \leftarrow \Enc(\sk_0, 0^L)$.
        \item $\cA$ stores $\ct^{1}_r$.
    \end{enumerate}
\end{enumerate}
\end{framed}
\caption{Recap of \cref{defn:AE-eval-security}}
\end{figure}
We use a standard hybridizing argument to show that the views of $\cA$ are indistinguishable. 
\begin{definition}
We define the Oracle $\widetilde{\Eval_{k, f}}(\sk_i, \cdot)$ as follows. For the first $k$ time it behaves exactly the same as $\Eval_f(\sk_i, \cdot)$. After $k$ calls, it outputs $\Enc(\sk_1, 0^L)$ for $\sk \leftarrow \gen(1^{\lambda})$.

For each $k \geq 0$, define $\Hyb_{\Eval, j}$ as the Real World Experiment $\Real_{\Eval}^{\cA}$ with $\Eval_f(\{\sk_i\}, \cdot)$ replaced by $\widetilde{\Eval_{k, f}}(\{\sk_i\}, \Eval, \cdot)$. 
\end{definition}
\begin{lemma}
     Assume $\Pi_{\ae}$ is IK-CCA secure. Then for any $0 \leq k \leq R$, we have 
    $$\Hyb_{\Eval, k + 1} \equiv_c \Hyb_{\Eval, k}.$$
\end{lemma}
\begin{proof}
    Suppose the contrary. Let $D$ be any distinguisher such that
    $$\PP(D(\Hyb_{\Eval, k + 1})) - \PP(D(\Hyb_{\Eval, k})) \geq \mathsf{poly}(\lambda)^{-1}.$$
    We design an adversary $\cA'$ to win the IK-CCA game. Let $i^* \in [N]$ be any index. On line (2) of $\mathbf{Exp}_{\mathsf{IK-CCA}}^{\cA'}$, the adversary $\cA'$ simulates $\Hyb_{\Eval, k}$ with $\sk_{i^*} = \sk_1$, and all other $\sk_i$ randomly generated. $\cA'$ simulates $\cA$ until the $k + 1$-th call to the oracle $\widetilde{\Eval_{j, f}}$. Let $i_{k + 1}, j_{k + 1}, \ct = \{\ct_{\ell}\}, \arg_p$ be the output of $\cA$. If $i_{k + 1} \neq i^*$, then $\cA'$ just guess randomly. Otherwise, $\cA'$ use the $\Dec$ oracle to compute $m'_\ell = \Dec(\sk_{j^*}, \ct_\ell)$. Then it outputs $m_0 = 0^L$, $m_1 = f(\{m'_\ell\}, \arg_p)$, and exits line (2) of $\mathbf{Exp}_{\mathsf{IK-CCA}}^{\cA'}$. Let $\ct$ be the output of line (5). $\cA'$ uses $\ct$ as the output of $\widetilde{\Eval_{k, f}}$, then continue to simulate $\Hyb_{\Eval, k}$ until the end. $\cA'$ returns $b' = 1$ iff $D$ accepts the resulting view.

    We condition on $i_{k + 1} = i^*$. If $b = 1$, then $\cA'$ perfectly simulates $\Hyb_{\Eval, k + 1}$, while if $b = 0$, then $\cA'$ perfectly simulates $\Hyb_{\Eval, k}$. Furthermore, in line (6) of \cref{defn:AE-IK-CCA}, $\cA'$ never use the $\Dec$ oracle, so $\ct_{\query} = \emptyset$. Therefore, we have
    \begin{align*}
     &\PP(\mathbf{Exp}_{\mathsf{IK-CCA}}^{\cA'} = 1) \\
     &= \frac{1}{2} + \PP(b' = 1, b = 1, i_{k + 1} = i^*) - \PP(b' = 1, b = 0, i_{k + 1} = i^*) \\   
     &= \frac{1}{2} + \frac{1}{2}\PP(D(\Hyb_{\Eval, k + 1}), i_{k + 1} = i^*)  \\
     & \quad - \frac{1}{2} \PP(D(\Hyb_{\Eval, k}), i_{k + 1} = i^*).
    \end{align*}
    By IK-CCA, there exists a negligible function $\mu(\lambda)$ independent of $j$ such that for any $j$, we have
    $$\PP(\mathbf{Exp}_\mathsf{IK-CCA}^{\cA'} = 1) \leq \frac{1}{2} + \mu(\lambda).$$
    Substituting back, and summing over all $i^* \in [N]$, we conclude a contradiction
    $$\PP(D(\Hyb_{\Eval, k + 1})) - \PP(D(\Hyb_{\Eval, k})) \leq \negl(\lambda).$$
\end{proof}
Since $\Hyb_{\Eval, 0}$ is identical to $\Ideal_{\Eval}^{\cA}$, and $\Hyb_{\Eval, R}$ is identical to $\Real_{\Eval}^{\cA}$, we conclude that $\Ideal_{\Eval}^{\cA}$ is indistinguishable with $\Real_{\Eval}^{\cA}$. So $\Pi_{\ae}$ is Eval-Secure as desired.

