\section{Proofs}
\label{sec:proofs}
In this section, we prove that our messaging scheme satisfies \cref{defn:messaging-correctness}, \cref{defn:messaging-security}, and \cref{defn:messaging-integrity} under the $B$-bounded friend scenario. These definitions establish correctness, security, and integrity, respectively.
\subsection{Proof of Correctness and Integrity}
We first show that $\Pi_{\asphr}$ satisfies both \cref{defn:messaging-correctness} and \cref{defn:messaging-integrity}.

We introduce some notations for convenience. Define
\begin{align*}
    \MSGS(t_0, i, j) = \{&(t, \msg): t \leq t_0 \land \\
    &\send(\reg_j, \msg) = \cI_{i, t} \land \\
             &\exists t' < t, \trust(\reg_j) = \cI_{i, t'}\}.
\end{align*}
Let $\msg_{ij}(\ell)$ be the $\ell$-th message in $\MSGS(t_0, i, j)$ sorted by timestep $t$. Let $\msg_{ij}^{lb}(\ell) = (\ell, \msg_{ij}(\ell)).$

\textbf{Consistent Prefix}: We show that $\Pi_{\asphr}$ satisfies \cref{defn:messaging-integrity}, and the consistent prefix property in \cref{defn:messaging-correctness}.
\begin{lemma}
\label{lem:messaging-correctness-main}
Consider the real-world experiment in \cref{expr:messaging-real-world}. Then with probability $1 - \negl(\lambda)$, for any pair of honest users $i \neq j \in \cH$ and any $t \leq T$, the property below holds.

\textbf{Property: }One of the following holds at the end of timestep $t$.

1) $\reg_j \notin C_i.\frienddb, C_i.\outb[\reg_j] = C_j.\inb[\reg_i] = \emptyset$.

2) $\reg_j \in C_i.\frienddb$. Then  for any $\ell$, $\msg_{ij}(\ell)$ is labeled with sequence number $\ell$. Furthermore, denote
\begin{align*}
S_t &= C_i.\frienddb[\reg_j].\seqs, \\
E_t &= C_i.\frienddb[\reg_j].\seqe, \\
R_t &= C_j.\frienddb[\reg_i].\seqr,
\end{align*}
(we define $R_t = 0$ if $\reg_i \notin C_j.\frienddb$). Then we have $S_t \in \{R_t, R_t + 1\}$, and
$$\abs{\MSGS(t, i, j)} = E_t,$$
$$C_i.\outb[\reg_j] = \{\msg^{lb}_{ij}(S_t), \cdots, \msg^{lb}_{ij}(E_t)\},$$
$$C_j.\inb[\reg_i] = \{\msg_{ij}(1), \cdots, \msg_{ij}(R_t)\}.$$
\end{lemma}
\begin{proof}
When $t = 0$, 1) is satisfied since $C_i.\frienddb$ is initialized as empty. We now show if these properties hold for all timesteps before $t$, then they will hold at the end of timestep $t$ with probability $1 - \negl(\lambda)$.

The relevant variables are only modified in Phase 1 of $C_i.\Userinput$, in $C_i.\Serverrpc$ and in $C_j.\Serverrpc$. We show that the lemma is satisfied after each method (no matter which order they execute).

Phase 1 of $C_i.\Userinput$
\hrule
If 1) holds before timestep $t$ starts, then unless 
$$\cI = \trust(\reg_j),$$
none of the variables are changed. Otherwise, we have $S_t = 1, E_t = R_t = 0$, and both $C_i.\outb[\reg_j] = C_j.\inb[\reg_i] = \emptyset$, which satisfies 2).

If 2) holds before timestep $t$ starts, then unless $\cI = \send(\reg_j, \msg)$ none of the variable are changed. Otherwise, $E_t = E_{t - 1} + 1$. We can check that
$$\MSGS(t, i, j) = \MSGS(t - 1, i, j) + (t, \msg).$$
Thus we have
$$\abs{\MSGS(t, i, j)} = \abs{\MSGS(t - 1, i, j)} + 1 = E_t.$$
Furthermore, we have $\msg = \msg_{ij}(E_t)$, and it will be labeled with sequence number $E_t$ by step (3) in $\Pi_{\asphr}.C.\mathsf{Input}()$ Phase 1. Thus $\msg^{lb} = \msg^{lb}_{ij}(E_t)$, so the equality with $C_i.\outb[\reg_j]$ is maintained. All the other variables remain unchanged. Thus the property remains true.

\vspace{10pt}
$C_i.\Serverrpc$
\hrule
The relevant variables only change during ACK decipher when a request is sent during Phase 2 of $C_i.\Userinput$. Let $j' = i_r$ be the PIR index chosen in that phase. No relevant variable changes if $j' \neq j$, so assume $j' = j$. We apply the EUF-CMA property of $\Pi_{\ae}$. With probability $1 - \negl(\lambda)$, the $\ack$ variable defined in step (7a) by
$$\ack = \Pi_{\ae}.\Dec(\sk_r, \ct_{\msg})$$
is either equal to $\bot$, or equal to a message user $j$ encrypted using $\sk_r$. Since user $j$ only encrypts messages and ACKS to user $i$ using $\sk_r$, we conclude that either $\ack = \ACK(R')$ for some $R' \leq R_{t - 1}$, or $\ack$ is equal to a previous labeled message user $j$ sent to user $j$. 

If $\ack$ is equal to a labeled message or $\bot$, client $i$ will skip steps (7c) and (7d), and no variable will be changed. Now consider the case
$$\ack = \ACK(R').$$
In step (7.c), we have $C_i.\frienddb[\reg_j].\seqs = S_{t - 1}$. By the induction hypothesis, we have
$$S_{t - 1} \geq R_{t - 1} \geq R'.$$
So no variable is changed unless $S_{t - 1} = R_{t - 1} = R'$, in which case $S_t = R_{t - 1} + 1$, and after popping $\msg^{lb}_{ij}(S_{t - 1})$ we get
$$C_i.\outb[\reg_j] = \{\msg^{lb}_{ij}(S_{t - 1} + 1), \cdots, \msg^{lb}_{ij}(E_t)\}.$$
Thus, the desired properties still hold.

\vspace{10pt}
$C_j.\Serverrpc$
\hrule
The relevant variables only change during step (6) (message decipher). Let $i' = i_r$ be the PIR index chosen in the previous call to $C_i.\Userinput$. No relevant variable changes if $i' \neq i$, so assume $i' = i$. By EUF-CMA, with probability $1 - \negl(\lambda)$, the deciphered message $\msg^{lb}$ is either $\bot$, or an ACK message, or a labeled message sent from user $i$ to user $j$ in a previous timestep $t' \leq t$.  In the first two cases, steps (6c) and (6d) are skipped, and no variable is updated. We now consider the last case. In this case, we have
$$\msg^{lb} = \begin{cases}
(-1, 0^{L_{\msg}}), S_{t' - 1} > E_{t'}. \\
\msg_{ij}^{lb}(S_{t' - 1}), S_{t' - 1} \leq E_{t'}. 
\end{cases}$$
Step (6c) and (6d) is not skipped only in the second case with $S_{t' - 1} = R_{t - 1} + 1$. By the induction hypothesis, in  this case we have $S_{t - 1} = S_{t' - 1} = R_{t - 1} + 1$. So $R_t = R_{t - 1} + 1 = S_{t - 1}$, and after popping $\msg_{ij}(S_{t - 1})$ we get
$$C_i.\outb[\reg_j] = \{\msg^{lb}_{ij}(1), \cdots, \msg^{lb}_{ij}(S_{t - 1})\}.$$
Thus, the desired properties still hold.

We have proven that the desired properties hold at the end of timestep $t$ with probability $1 - \negl(\lambda)$.
\end{proof}
We now show that $\Pi_{\asphr}$ satisfies \cref{defn:messaging-integrity}. We use the notations in \cref{lem:messaging-correctness-main}. Let $(\cF, \cM) = C_j.\mathsf{GetView}()$. Take $t(j) = T_0$. Unpacking the definition of $\cF$ and $\cM$, we need to verify that for each $i \in [N]$, with probability $1 - \negl(\lambda)$ there exists a $t(i) \leq T_0$ such that
 
 \begin{multline*}
 C_j.\inb[\reg_i] = \{\msg: \exists t \leq t(i), \send(\reg_j, \msg) = \cI_{i, t} \land \\
 \exists t' < t, \trust(\reg_j) = \cI_{i, t'} \land \\
 \exists t'' \leq T_0, \trust(\reg_i) = \cI_{j, t''}\}.   
 \end{multline*} 
 %That a) holds is easy to see, since by the protocol definition, $\reg_i$ is inserted into $C_j.\frienddb$ during timestep $t$ if and only if $\trust(\reg_i) = \cI_{j, t}$, and it is never removed. 
  If $i = j$, then the equation holds for $t(j) = T_0$ since messages to oneself are deposited to $C_j.\inb$ immediately in Phase 1 of $\Pi_{\asphr}.C.\mathsf{input}$. Now assume $i \neq j.$ 
 
 We show that $t(i)$ exists as long as \cref{lem:messaging-correctness-main} holds at the current timestep $T_0$. We consider which scenario in the lemma holds. 
 
 If 1) holds, let $t(i) = T_0$. In this case, both sides of the equation are $\emptyset$.
 
 If 2) holds, let $t(i) = T_0$ if $\reg_i \notin C_j.\frienddb$ at the current timestep. Otherwise, let $t(i)$ be the timestep before $C_i.\send(\reg_j, \msg_{ij}(R_{t} + 1))$ is called(or the current timestep if $\msg_{ij}(R_{t} + 1)$ does not exist). If $\reg_i \notin C_j.\frienddb$, we have $R_t = 0$, so both sides of the equation are empty sets. Otherwise, by the definition of $\msg_{ij}(\ell)$, both sides of the equation are equal to $\{\msg_{ij}(1),\cdots, \msg_{ij}(R_t)\}$. So the equality in \cref{defn:messaging-integrity} holds when the property in \cref{lem:messaging-correctness-main} holds for all pair of honest users $i, j \in \cH$. Thus, \cref{lem:messaging-correctness-main} implies \cref{defn:messaging-integrity}.

The consistent prefix property in \cref{defn:messaging-correctness} is an easy corollary of \cref{defn:messaging-integrity} when the server behaves honestly.
 
 \textbf{Eventual Consistency}: We now show the eventual consistency property in \cref{defn:messaging-correctness}. Take the same choice of $t(i)$ as in the previous subsection. Let $T_{cons} = 2\lambda B^2 T^U_{\mathsf{trans}} \cdot T_1 + T_1.$ We show that this $T_{cons}$ satisfies the desired property. 
 
 We wish to show that if $T_0 \geq T_{cons}$, then with probability $1 - \negl(\lambda)$ we have $t(i) \geq T_1$. Let $E = \abs{\MSGS(T_1, i, j)}$(Note $E$ is independent of the protocol execution). If $R_{T_1} \geq E$, then by casework on the definition we always have $t(i) \geq T_1$. So it suffice to show that $R_{T_1} < R$ with negligible probability. 
 
 We use the same notation as \cref{lem:messaging-correctness-main}. For each $k \leq E$, let $X_k$ be the random variable denoting the first $t$ such that $R_t \geq k$, with $X_0 = 0$. Let $Y_k$ denote the first $t$ such that $S_{t} > k$. It suffices to show that for any $k \leq E$, we have
 $$Y_k - X_k > \lambda B^2 T^U_{\mathsf{trans}}$$
 or
 $$X_{k} - \max(Y_{k - 1}, T_1) > \lambda B^2 T^U_{\mathsf{trans}}$$
 with negligible probability. For each timestep $t$ between $X_k$ and $X_{k + 1}$ such that $t$ divides $C_i.T_{\mathsf{trans}}$, let $j_t$ denote the $i_r$ that $C_i$ chooses, and let $i_t$ denote the $i_s$ that $C_j$ chooses for their last non-empty request to the server before timestep $t + 1$. If $(i_t, j_t) = (i, j)$, we must have $S_{t} \geq k + 1$ by the protocol definition since $C_i$ reads the ACK message $C_j$ deposited to the server database. Furthermore, if we let $K = \ceil{T^U_{\mathsf{trans}} / C_i.T_{\mathsf{trans}}}C_i.T_{\mathsf{trans}}$, then the random variables
 $$(i_t, j_t), (i_{t + K}, j_{t + K}), (i_{t + 2K}, j_{t + 2K})\cdots$$
are mutually independent since client $i$ and client $j$ both made a non-empty request to the server during timestep $(t, t + K]$. Therefore, the probability that $Y_k - X_k > \lambda B^2 T^U_{\mathsf{trans}}$ is at most
$$(1 - B^{-2})^{\lambda B^2 T^U_{\mathsf{trans}} / K} = \negl(\lambda)$$
as desired. Analogous, the event $X_{k} - \max(Y_{k - 1}, T_1) > \lambda B^2 T^U_{\mathsf{trans}}$ is also negligible. Thus, eventual consistency holds for $\Pi_{\asphr}$.
\subsection{Proof of Security}
In this section we show that $\Pi_{\asphr}$ satisfies \cref{defn:messaging-security}. 

Let $N'(\lambda) = N(\lambda)^2$ and $R'(\lambda) = 2N'(\lambda)T(\lambda)$. Assume the simulator $\Sim_{\sym}$ satisfy \cref{defn:AE-eval-security} with parameters $(N, R)$ equal to $(N'(\lambda), R'(\lambda))$, and let $\Sim_{\pir}$ be a simulator satisfying \cref{defn:PIR-SIM-security} with parameters $(N, R) = (N(\lambda), R'(\lambda))$. 

At step (2), (4.a), and (4.c) of \cref{expr:messaging-ideal-world}, the simulator $\Sim_{\asphr}$ runs modified versions of the client methods in the corresponding steps of \cref{expr:messaging-real-world} for each client $i \in \cH$. Modifications are marked with a sideline. 

For each pair of honest users $i, j \in \cH$, let 
$$(\sk_{ij, r}, \sk_{ij, w}) = \gensharedsecret(\reg_i, \reg_j)$$ 
be their shared secret, with $\sk_{ij, r} = \sk_{ji, w}$ being the read key of user $i$ for messages from user $j$.

%\todo{Hard-coded. Please modify if the $\Pi_{\asphr}$ methods get changed.}

\vspace{10pt}

$\mathbf{\Sim_{\asphr}.\Register}(1^{\lambda}, i, N)$
\vspace{5pt}
\hrule
\vspace{5pt}
No modification.
\vspace{10pt}

$\mathbf{\Sim_{\asphr}.\Userinput}(t, i, { \textbf{Hybrid 3:} \text{replace $\cI$ with $\Leak$}})$
\vspace{5pt}
\hrule
\vspace{5pt}
Phase 1: 

\begin{siderule}
{
 

\textbf{Hybrid 3}:

Initialize $\cI$.

\begin{enumerate}
    \item If $(i, \reg, t) \in \Leak_f$, $\cI \leftarrow \trust(\reg)$.
    \item If $(i, \reg, \msg, t) \in \Leak_m$, $\cI \leftarrow \send(\reg, \msg)$.
    \item Else, $\cI \leftarrow \emptyset$.
\end{enumerate}

}
\end{siderule}

If $\cI = \emptyset$, do nothing. 

If $\cI = \send(\reg, \msg)$, 

\begin{enumerate}
    \item Check that $\reg$ is in $\frienddb$. If not, skip to Phase 2.
    \item If $\reg$ is the registration of $C_i$ itself, append $\msg$ to $\inb[\reg]$, and skip to Phase 2. \footnote{Skipping this step breaks consistent prefix.}
    \item Add $1$ to $\frienddb[\reg].\seqe$. 
    \item Push $\msg^{lb} = (\frienddb[\reg].\seqe, \msg)$ to $\outb[\reg]$.
\end{enumerate}

If $\cI = \trust(\reg)$.
\begin{enumerate}
    \item Check if $\reg$ is in $\frienddb$. If so, skip to Phase 2.
    \item $\sk_r, \sk_w \leftarrow \gensharedsecret(\reg_m, \reg).$ Here $\reg_m$ is the user's own registration information.
    \item $\frienddb[\reg] \leftarrow \{\sk: (\sk_r, \sk_w),  \seqs: 1, \seqe: 0, \quad \\ \seqr: 0\}$.
\end{enumerate}

Phase 2:
\begin{enumerate}
    \item If $t$ is not divisible by $T_{\trans}$, return $\emptyset$.
    \item Let $\{\reg_1, \cdots, \reg_k\}$ be the keys of $\frienddb$, with $k\leq B$. Construct $S = [\reg_1,\cdots, \reg_k, \reg, \cdots, \reg]$, where we add $B - k$ copies of $\reg = (-1)$, a dummy registration info . Sample $\reg_s, \reg_r$ uniformly and independently at random from $S$. 
    \item Let $\msg$ be the (labeled) message with sequence number $\frienddb[\reg_s].\seqs$ in $\outb[\reg_s]$. If $\outb[\reg_s]$ is empty, let $\msg = (-1, 0^{L_{\msg}})$.
    \item $\sk_w \leftarrow \frienddb[\reg_s].\sk[1]$. If $\reg_s$ does not exist in $\frienddb$, randomly generate a secret key $\sk \leftarrow \Pi_{\ae}.\gen(1^{\lambda}).$
    \item $\seqr \leftarrow \frienddb[\reg_s].\seqr$.
    \item Encrypt Messages with $\sk$.

    \begin{siderule}
        { 
        \textbf{Hybrid 1}:
        
        If $\reg_s \in \reg_\cH$, 
        \begin{itemize}
            \item $\ct_{\msg} \leftarrow \Sim_{\sym}(1^{\lambda}),$
            \item $\ct_{\ack} \leftarrow \Sim_{\sym}(1^{\lambda}),$
        \end{itemize}
        Else, 
        }
    \end{siderule}
    
    
    \begin{itemize}
        \item $\ct_{\msg} = \Pi_{\ae}.\enc(\sk_w, \msg).$
        \item $\ct_{\ack} = \Pi_{\ae}.\enc(\sk_w, \ACK(\seqr))$.
    \end{itemize}
    \item Let $\reg_r = (i_r, \_)$. Formulate a PIR request for index $i_r$. 

     \begin{siderule}
     {
     
          
     \textbf{Hybrid 2}:
         
     If $\reg_r \in \reg_{\cH}$, 
    \begin{itemize}
        \item $\ct_{\query} \leftarrow \Sim_{\pir}(1^{\lambda}).$
    \end{itemize}
    Else, 
    }
    \end{siderule}
    \begin{itemize}
        \item $\ct_{\query}, \sk_{\pir} \leftarrow \Pi_{\pir}.\query(1^{\lambda}, i_r).$
    \end{itemize}
    \item return $\req = (\ct_{\msg}, \ct_{\ack}, \ct_{\query})$.
    \item Remember $\reg_r$.
\end{enumerate}
\vspace{10pt}
$\mathbf{\Pi_{\asphr}.C.\Serverrpc}(t, \resp)$
\vspace{5pt}
\hrule
\vspace{5pt}
\begin{siderule}
{ \textbf{Hybrid 1'}: if $\reg_r \in \reg_{\cH}$, return.}
\end{siderule}
\begin{enumerate}
    \item If $\resp = \emptyset$, return.
    \item Parse $\resp = (a_{\msg}, a_{\ack})$. Let $\reg_r, \sk_{\pir}$ be defined in the last call to $\Pi_{\asphr}.C.\Userinput$.
    \item $\ct_{\msg} \leftarrow \Pi_{\pir}.\Dec(1^{\lambda}, \sk_{\pir}, a_{\msg}).$
    \item $\ct_{\ack} \leftarrow \Pi_{\pir}.\Dec(1^{\lambda}, \sk_{\pir}, a_{\ack}).$
    \item $\sk_r \leftarrow \frienddb[\reg_r].\sk[0]$.
    \item Decipher the message.
    \begin{enumerate}
        \item $\msg^{lb} \leftarrow \Pi_{\ae}.\Dec(\sk_r, \ct_{\msg})$.
        \item If $\msg^{lb} = \bot$ or $\msg^{lb}[0]$ is not $\frienddb[\reg_r].\seqr + 1$, ignore the message.
        \item Add $1$ to $\frienddb[\reg_r].\seqr$. 
        \item Let $\msg$ be $\msg^{lb}[1]$. Push $\msg$ to $\inb[\reg_r]$.
    \end{enumerate}
    \item Decipher the ACK.
    \begin{enumerate}
        \item $\ack \leftarrow \Pi_{\ae}.\Dec(\sk_r, \ct_{\ack})$.
        \item If $\ack = \bot$ or $\ack$ is not the form $\ACK(k)$ for some $k$, ignore the ack.
        \item Let $\ack = \ACK(k)$. If $k < \frienddb[\reg_r].\seqs$, ignore the ack.
        \item $\frienddb[\reg_r].\seqs \leftarrow k + 1$. Remove the message with sequence number $k$ from $\outb[\reg_r]$.
    \end{enumerate}
\end{enumerate}

We use a hybrid argument to show that the two views are indistinguishable. We start from the original implementation of the client methods $\Pi_{\asphr}.C$, and use a hybrid argument to transform it into the simulator methods $\Sim_{\asphr}$. We call the Real World Experiment $\Hyb_0$.  

% The todo's below have been resolved.

%\todo{The first hybrid here uses a very strong $\Eval$ oracle. The reason is the following: if an adversary $\cA$ exists that breaks indistinguishability, then we want to construct an adversary $\cA_1$ that breaks our security definition \cref{defn:AE-eval-security}. $\cA_1$ has to "simulate" the view of $\cA$ while only knowing the public key of the adversaries. Now assume the adversary $\cA$ establishes trust with a certain malicious public key $k$. Then $\cA_1$ must be able to accurately simulate all the internal state changes related to this public key, since the associated ACKs and messages are contained in the view of $\cA$. On the other hand, $\cA_1$ cannot possibly do this with only the encryption oracle, since $k$ might not have a computable private key. So the Eval oracle comes to the rescue.}

%\todo{If we can somehow bypass this issue, the length of the paper can be reduced by at least 5 pages. So please share any possible ideas.}

First Hybrid: We add the statements marked Hybrid 1 and run the experiment in \cref{defn:messaging-real-world-experiment}. We call this modified experiment $\Hyb_1$. To argue this preserves indistinguishability, suppose on the contrary that an adversary $\cA$ and a distinguisher $\cD$ can distinguish the view before and after the modification. Then we can build an adversary $\cA_1^O$ and a distinguisher $\cD'$ breaking \cref{defn:AE-eval-security}. 

The adversary $\cA_1^O$ simulates the real-world experiment in \cref{expr:messaging-real-world}. tTe key idea is to choose a powerful function $f: \Sigma^* \times \Sigma^* \to \Sigma^L$. For each pair of honest users $i, j \in \cH$, define $\data_r[\reg_i] = (\frienddb[\reg_i], \inb[\reg_i], \outb[\reg_i])$. $\cA_1$ stores a log of changes to $\data$ \textbf{encrypted using $\sk = \sk_{ij, r}$}. Whenever it wants to access or update these data in the client simulation, it calls $\Eval_f$. We choose $f$ to decrypt the log, recover the plaintext of the fields, do the corresponding simulation, then re-encrypt the outputs and update to the log ($f$ can use $\arg_p$ to determine which line it is on). For such an $f$, $\cA_1^O$ can use $f$ to perfectly simulate client updates on the $\sk$-encrypted $\data[\reg]$. Finally, note that any client outputs computed from $\data$ are encrypted using $\sk_{ij, r}$, so $f$ can also perfectly simulate client outputs. 

With this choice of $f$, we describe $\cA_1^O$. For any $1 \leq i, j \leq N$, denote $\sk_{ij, r} = \sk_{ji, w} = \sk_{i * N + j}$, where $\sk_{*}$ are defined line (1) of \cref{defn:AE-eval-security}. The step numbers below refer to the steps in \cref{defn:messaging-real-world-experiment} unless otherwise indicated. 
\begin{itemize}
    \item Simulate step (1), (2), (3), and (4.b) identical to \cref{defn:messaging-real-world-experiment}.
    \item After step (2), for each pair of $i, j \in \cH$, $\cA_1$ "set"\footnote{In other word, $\cA_1$ simulates the honest users as if $\gensharedsecret(\reg_i, \reg_j) = \sk_{ij}$. $\cA_1$ does not have access to $\sk_{ij}$.} 
    $$\gensharedsecret(\reg_i, \reg_j) = (\sk_{ij, r}, \sk_{ji, w}).$$ 
    $\cA_1$ independently generates all other shared secrets using $\Pi_{\ae}.\gen(1^{\lambda}).$
    \item On step (4.a), $\cA_1$ iterates over $i \in \cH$. $\cA_1$ simulates client $i$'s action in $\Pi_{\asphr}.\Userinput$ verbatim until Phase 2 Step (6). In Step (6), $\cA_1$ do casework based on if $\reg_s \in \reg_{\cH}$. 
    \begin{itemize}
        \item  If $\reg_s \notin \reg_{\cH}$, $\cA_1$ knows the shared secret between $\reg_i$ and $\reg_s$, so $\cA_1$ can simulate this step verbatim.
        \item  If $\reg_s \in \reg_{\cH}$, assume $\reg_s = \reg_j$ where $j \in \cH$. $\cA_1$ sets $(\ct, \arg_p)$ so that $\Eval_f(\sk_{ij}, \ct, \arg_p)$ can simulate the original step (6) to compute $\ct_{\msg}$, outputs $(i, j, \ct, \arg_p)$, then moves to line (4.b) in \cref{defn:AE-eval-security}. $\cA_1$ sets $\ct_{\msg} = \ct^b_r$ to be the output of line (4.b). $\cA_1$ repeats this for $\ct_{\ack}$.
    \end{itemize}
   
    \item On step (4.c), $\cA_1$ simulates client action $\Pi_{\asphr}.\Serverrpc$ using $\Eval_f$.
\end{itemize}
We note that in $\Real_{\Eval}^{\cA}$, $\cA_1$ perfectly simulates the view of $\cA$ in $\Hyb_0$. In $\Ideal_{\Eval}^{\cA, \Sim_{\ae}}$, $\cA_1$ perfectly simulates the view of $\cA$ in $\Hyb_1$. Thus, $\cA_1$ just need to output the view of $\cA$ at the end of the simulation, and the same distinguisher $\cD' = \cD$ can distinguish between the views of $\cA_1$ in the real and ideal world. This contradicts our assumption that $\Pi_{\ae}$ satisfies \cref{defn:AE-eval-security}. Therefore, we finally conclude that
$$\Hyb_0 \equiv_c \Hyb_1.$$

Corollary of First Hybrid: We add the statements marked Hybrid 1' and run the experiment in \cref{defn:messaging-real-world-experiment}. We call this modified experiment $\Hyb_{1'}$.

We argue that this hybrid does not change the adversary's view at all. $\Pi_{\asphr}.C.\Serverrpc$ only updates fields of $\data[\reg_r]$. In $\Hyb_1$, for any $\reg \in \reg_{\cH}$, the fields of $\data[\reg]$ does not affect the client's output. So adding the statement' Hybrid 1' does not affect the client's output. We conclude that
$$\Hyb_1 \equiv_c \Hyb_{1'}.$$

Second Hybrid: We add the statements marked Hybrid 2 and run the experiment in \cref{defn:messaging-real-world-experiment}. We call this modified experiment $\Hyb_{2}$.

We argue by contradiction that $\Hyb_{1'}$ and $\Hyb_{2}$ are indistinguishable. Suppose that an adversary $\cA$ and a distinguisher $\cD$ can $\Hyb_{1'}$ and $\Hyb_{2}$. Then we can build an adversary $\cA_2$ and a distinguisher $\cD'$ breaking \cref{defn:PIR-SIM-security}. 

$\cA_2$ simulates a modified version of $\Hyb_1$. $\cA_2$ simulates line (1, 2, 3, 4b, 4c) in \cref{defn:messaging-real-world-experiment} verbatim as in both $\Hyb_{1'}$ and $\Hyb_2$. On line (4a), $\cA_2$ simulates everything but Phase 2 step (7) verbatim. When it reaches Phase 2 step (7), if $\reg_r \notin \reg_{\cH}$ then it simulates this step verbatim as well. If $\reg_r \in \reg_{\cH}$, then $\cA_2$ returns $i_r$ and exits line (1a) of the experiment in \cref{defn:PIR-SIM-security}. Let $\ct^b_r$ be the return value of line (1b). $\cA_2$ sets $\ct_{\query} = \ct^b_r$ and continues simulation.

Contrary to the first hybrid, $\cA_2$ knows all the key exchange secret keys. Furthermore, if $\reg_r \notin \reg_{\cH}$, $\cA_2$ knows $\sk_{\pir}$ generated by $\Pi_{\asphr}.C.\Userinput$. If $\reg_r \notin \reg_{\cH}$, then $\cA_2$ does not know the $\sk_{\pir}$, but this $\sk_{\pir}$ will never be used since Hybrid 1' ensures that $\Pi_{\asphr}.C.\Serverrpc$ is skipped. Thus, we conclude that $\cA_2$ can complete the simulation.

In $\Real_{\pir}^{\cA}$, $\cA_2$ simulates $\Hyb_{1'}$ verbatim, while in $\Ideal_{\pir}^{\cA, \Sim}$, $\cA_2$ simulates $\Hyb_2$ verbatim. Thus, $\cA_2$ just need to output the view of $\cA$ at the end of the simulation, and the same distinguisher $\cD' = \cD$ can distinguish between the views of $\cA_1$ in the real and ideal world. This contradicts our assumption that $\Pi_{\pir}$ satisfies \cref{defn:PIR-SIM-security}. Therefore, we conclude that
$$\Hyb_{1'} \equiv_c \Hyb_2.$$
Third Hybrid: We add the statements marked Hybrid 3, and run the experiment in \cref{defn:messaging-real-world-experiment}. We call this modified experiment $\Hyb_{3}$. This modification does not change the view of the adversary at all. After the changes in Hybrid 1, 1', and 2, for any $i, j \in \cH$, the contents of $C_i.\frienddb[\reg_j], C_i.\inb[\reg_j], C_i\outb[\reg_j]$ does not affect client $i$ output. (In particular, whether $\reg$ lies in $\frienddb$ or not does not affect the distribution of $\reg_s$ or $\reg_r$). For any $\reg \notin \reg_{\cH}$, Hybrid 3 does not affect updates to $C_i.\frienddb[\reg], C_i.\inb[\reg]$, and $C_i\outb[\reg]$. So we conclude
$$\Hyb_{2} \equiv_c \Hyb_3.$$
Finally, we conclude that
$$\Hyb_0 \equiv_c \Hyb_3$$
Note that $\Hyb_0 = \Real_{\msg}^{\cA}$ and $\Hyb_3 = \Ideal_{\msg}^{\cA, \Sim_{\asphr}}$,. Therefore, $\Pi_{\asphr}$ satisfies \cref{defn:messaging-security}.
