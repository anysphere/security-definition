\section{Introduction}
\label{sec:purpose}
This paper is motivated by our experience of developing a metadata-private messaging (MPM) service called Anysphere. In our whitepaper \cite[Section 3]{whitepaper}, we describe Anysphere's threat model and core protocol at a high level. The original intent of this document was to rigorously show that Anysphere’s core protocol satisfies the metadata privacy we promise. In the process, we discovered the need for a new security definition, which may be of independent interest.

Existing security proofs of MPM systems (such as \cite{corrigan2010dissent, corrigan2015riposte, angel2016unobservable, ahmad2021addra}) have shown the privacy of a private-information-retrieval (PIR) system where users can deposit and retrieve information without revealing metadata to the server. We found that we wanted stronger guarantees for several reasons.
\begin{itemize}
    \item The security of the PIR system does not guarantee the security of the messaging system as a whole. A well-known example illustrating this is the compromised-friend (CF) attack proposed by Angel, Lazar and Tzialla (\cite{angel2018cf}). They show that if an honest user befriends a malicious user, then the metadata of conversations between two honest users might be compromised even with a secure PIR system. To our knowledge, no proofs exist that show immunity against CF attacks\footnote{Pung's security proof \cite{angel2018thesis} assumes honest users only ever talk to honest users.}. 
    
    We found a more powerful CF attack, described in \cref{sec:security-vulnerable}, while writing this paper. This vulnerability affects existing implementations of Pung and Addra, and can result in full compromise of metadata (timing, sender and receiver of PIR queries). This attack demonstrates that CF attacks can potentially cause more damage than previously expected \cite{angel2018cf}. In light of this attack, we wish to rigorously prove that our MPM system can defend against CF attacks.
    
    \item Anysphere is based on Addra \cite{ahmad2021addra}. Addra is originally designed for users to hold exactly one conversation at a time. In our application, clients may hold many different conversations at the same time. We need to ensure that our adaptation does not introduce new vulnerabilities.
    
    \item Addra and Pung \cite{angel2016unobservable} assume that clients run in synchronous round, and each client sends exactly one message to the server each round. As clients have different level of resources, running synchronous rounds is not economical. For example, corporate users might wish rounds run faster to receive timely updates, while individual users might not want to participate in each round to preserve bandwidth. Anysphere uses asynchronous rounds where each client can transmit on a different schedule. We need to justify the security definition to incorporate this.
    
    \item The MPM systems mentioned above lack a mechanism to detect and retransmit lost or shuffled packets. We introduce an acknowledgement (ACK) mechanism in Anysphere to fix this issue. The security of this mechanism cannot be justified by previous security definitions. This is because the previous security definitions assume that the clients do not change their behavior based on the server's response, which is inherently not true for messaging systems.
\end{itemize}

This paper is organized as follows. In \cref{sec:general-defn}, we present a formal security definition of what it means for a whole messaging system to be correct and secure under our threat model outlined in \cite{whitepaper}. In \cref{sec:security-vulnerable}, we describe the new CF attack, which we name the \textit{PIR replay attack}. In \cref{sec:asphr-defn}, we describe a slightly modified Anysphere core protocol in pseudocode. In \cref{sec:proofs} we prove that the protocol defined in \cref{sec:asphr-defn} satisfies the security definition in \cref{sec:general-defn}. A technical detail in the proof is delayed to \cref{sec:IND-CCPKA}. In \cref{sec:actual-asphr-protocol}, we describe and prove the security of Anysphere's real-world implementation. We focus on a technique we use called prioritization, which trades a small metadata leakage for efficiency.

%\todo{Currently, we assume all clients have registered before execution. We also do not handle trust establishment.}

%\todo{Also, CF attack is a huge problem! What should we do about it?}

\section{Conventions}
We use the following notational conventions.
\begin{itemize}
    \item When we write $f(\cdot)$, the dot might hide several variables.
    
    \item Given an oracle $O(x, \cdot)$ and a series $\{x_i\}$, we define \\ $O(\{x_i\}, \cdot)$ as the oracle whose input contains an extra argument $j$ and outputs $O(\{x_i\}, j, \arg) = O(x_j, \arg)$.

    \item Given a series $\{a_i\}$, and a set of indices $S$, let $a_S$ denote $\{a_i\}_{i \in S}$. 

    \item Experiments are always parametrized by the security parameter $\lambda$. When we say two experiments are indistinguishable, we mean the views of the adversary in the two experiments are indistinguishable. The view of the adversary consists of all inputs, outputs, and internal randomness of the adversary.
    
    \item When machines ``return'' in a method, they do not execute any subsequent commands and exit the method immediately.
\end{itemize}
